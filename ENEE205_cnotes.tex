\documentclass[10pt,landscape]{article}
\usepackage{multicol}
\usepackage{calc}
\usepackage{ifthen}
\usepackage[landscape]{geometry}
\usepackage{amsmath,amsthm,amsfonts,amssymb}
\usepackage{color,graphicx,overpic}
\usepackage{hyperref}
\usepackage{siunitx}


\pdfinfo{
  /Title (example.pdf)
  /Creator (TeX)
  /Producer (pdfTeX 1.40.0)
  /Author (Seamus)
  /Subject (Example)
  /Keywords (pdflatex, latex,pdftex,tex)}

% This sets page margins to .5 inch if using letter paper, and to 1cm
% if using A4 paper. (This probably isn't strictly necessary.)
% If using another size paper, use default 1cm margins.
\ifthenelse{\lengthtest { \paperwidth = 11in}}
    { \geometry{top=.5in,left=.5in,right=.5in,bottom=.5in} }
    {\ifthenelse{ \lengthtest{ \paperwidth = 297mm}}
        {\geometry{top=1cm,left=1cm,right=1cm,bottom=1cm} }
        {\geometry{top=1cm,left=1cm,right=1cm,bottom=1cm} }
    }

% Turn off header and footer
\pagestyle{empty}

% Redefine section commands to use less space
\makeatletter
\renewcommand{\section}{\@startsection{section}{1}{0mm}%
                                {-1ex plus -.5ex minus -.2ex}%
                                {0.5ex plus .2ex}%x
                                {\normalfont\large\bfseries}}
\renewcommand{\subsection}{\@startsection{subsection}{2}{0mm}%
                                {-1explus -.5ex minus -.2ex}%
                                {0.5ex plus .2ex}%
                                {\normalfont\normalsize\bfseries}}
\renewcommand{\subsubsection}{\@startsection{subsubsection}{3}{0mm}%
                                {-1ex plus -.5ex minus -.2ex}%
                                {1ex plus .2ex}%
                                {\normalfont\small\bfseries}}
\makeatother

% Define BibTeX command
\def\BibTeX{{\rm B\kern-.05em{\sc i\kern-.025em b}\kern-.08em
    T\kern-.1667em\lower.7ex\hbox{E}\kern-.125emX}}

% Don't print section numbers
\setcounter{secnumdepth}{0}


\setlength{\parindent}{0pt}
\setlength{\parskip}{0pt plus 0.5ex}

%My Environments
\newtheorem{example}[section]{Example}
% -----------------------------------------------------------------------

\begin{document}
\raggedright
\footnotesize
\begin{multicols}{3}


% multicol parameters
% These lengths are set only within the two main columns
%\setlength{\columnseprule}{0.25pt}
\setlength{\premulticols}{1pt}
\setlength{\postmulticols}{1pt}
\setlength{\multicolsep}{1pt}
\setlength{\columnsep}{2pt}

\begin{center}
     \Large{\underline{ENEE 205}} \\
\end{center}

\section{Midterm  Examination 1}

\subsection{1.1 Basic concepts}
\begin{enumerate}
	\item Electric charge $q$, $1e^- = \SI{1.6e-19}{\coulomb}$
	\item Electric field $\vec{E}=\frac{1}{r\pi\epsilon_0}\frac{q_1}{r^2}\hat{r}$
	\item Electric force on point charge $\vec{F}=K\frac{q_1q_2}{r^2}=\vec{E}q_2$
	\item Electric current $i = \frac{\mathrm{d}q}{\mathrm{d}t}$
	\item Voltage $\int_{x_1}^{x_2}E\mathrm{d}x=v(2)-v(1)=v_{21}$
	\item Resistance $R=\frac{L}{\sigma A}=\rho\frac{L}{A}$
	\item Resistance from resistivity: $R=\frac{\rho l}{A}$
	\item Capacitance $C = \epsilon \frac{A}{d}$
	\item Parallel plates: $C = \epsilon \frac{A}{d}$
	\item Inductance $L=\mu N^2 \frac{A}{d}$
	\item Solenoid: $ L = \mu N^2\frac{A}{d}$
	\item Ohm's law $v=iR$
	\item TR: Capacitors $i=C\frac{\mathrm{d}v}{\mathrm{d}t}$
	\item TR: Inductors $v=L\frac{\mathrm{d}i}{\mathrm{d}t}$
	\item Impedance: Inductors $Z=jL\omega$
	\item Impedance: Inductors $Z=jL\omega$
	\item Impedance: Capacitors $Z=\frac{1}{jC\omega}$
	\item Voltage Divider (Only in Series): $V_Z1=VT\frac{Z1}{ZT}$
	\item Current Divider (Only in Parallel): $I_Z1=IT\frac{Z1}{ZT}$

	
\end{enumerate}

\subsection{1.2 Moar basic concepts}
Total charge accumulated over time is $q=\int_0^t i(t) \mathrm{d}t $ \\
Units of current are \si{C.s^{-1}}\\
Instantaneous power is defined as $p(t) = v(t)i(t)$\\
Current flows from higher to lower potential\\
Ideal voltage source ALWAYS puts out the same voltage for any (or even no) load.\\
Ideal current source ALWAYS puts out the same current for any (or even infinite) load.\\
Open circuit:\\ %diagram here
Closed circuit:\\

\subsection{1.3 Circuit Analysis: Kirchhoff's Laws} %todo: add diagrams
Branch: a line with one two-terminal device\\
Node: where two or more branches meet\\
	\hspace{4ex} Trivial node: only two connected elements\\
	\hspace{4ex} Nontrivial node: more than two connected elements\\
Loop: branches that form a closed path\\
Mesh: Loop that does not surround any branches\\
	\hspace{4ex} Trivial mesh: contains only two elements\\
	\hspace{4ex} Nontrivial mesh: contains more than two elements\\
	
%diagram with nodes, meshes, and branches here

Kirchhoff's current law: the sum of currents at every node is 0\\

\subsection{1.4 Moar circuit analysis with Kirchhoff's Laws}
You can solve a system of linear equations derived from Kirchoff's laws with a system of matricies\\

\subsection{1.5 Sinusoidal Functions}
AC involves 3 parameters: amplitude, angular frequency, and phase.\\
This means remember how to use those trig identities\\



\section{Section 3}
Etc.

% You can even have references
\rule{0.3\linewidth}{0.25pt}
\scriptsize
\bibliographystyle{abstract}
\bibliography{refFile}
\end{multicols}
\end{document}